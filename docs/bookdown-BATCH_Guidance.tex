\documentclass[]{book}
\usepackage{lmodern}
\usepackage{amssymb,amsmath}
\usepackage{ifxetex,ifluatex}
\usepackage{fixltx2e} % provides \textsubscript
\ifnum 0\ifxetex 1\fi\ifluatex 1\fi=0 % if pdftex
  \usepackage[T1]{fontenc}
  \usepackage[utf8]{inputenc}
\else % if luatex or xelatex
  \ifxetex
    \usepackage{mathspec}
  \else
    \usepackage{fontspec}
  \fi
  \defaultfontfeatures{Ligatures=TeX,Scale=MatchLowercase}
\fi
% use upquote if available, for straight quotes in verbatim environments
\IfFileExists{upquote.sty}{\usepackage{upquote}}{}
% use microtype if available
\IfFileExists{microtype.sty}{%
\usepackage{microtype}
\UseMicrotypeSet[protrusion]{basicmath} % disable protrusion for tt fonts
}{}
\usepackage[margin=1in]{geometry}
\usepackage{hyperref}
\hypersetup{unicode=true,
            pdftitle={Planning Tool Guidance},
            pdfauthor={Project Big Life},
            pdfborder={0 0 0},
            breaklinks=true}
\urlstyle{same}  % don't use monospace font for urls
\usepackage{natbib}
\bibliographystyle{apalike}
\usepackage{color}
\usepackage{fancyvrb}
\newcommand{\VerbBar}{|}
\newcommand{\VERB}{\Verb[commandchars=\\\{\}]}
\DefineVerbatimEnvironment{Highlighting}{Verbatim}{commandchars=\\\{\}}
% Add ',fontsize=\small' for more characters per line
\usepackage{framed}
\definecolor{shadecolor}{RGB}{248,248,248}
\newenvironment{Shaded}{\begin{snugshade}}{\end{snugshade}}
\newcommand{\KeywordTok}[1]{\textcolor[rgb]{0.13,0.29,0.53}{\textbf{#1}}}
\newcommand{\DataTypeTok}[1]{\textcolor[rgb]{0.13,0.29,0.53}{#1}}
\newcommand{\DecValTok}[1]{\textcolor[rgb]{0.00,0.00,0.81}{#1}}
\newcommand{\BaseNTok}[1]{\textcolor[rgb]{0.00,0.00,0.81}{#1}}
\newcommand{\FloatTok}[1]{\textcolor[rgb]{0.00,0.00,0.81}{#1}}
\newcommand{\ConstantTok}[1]{\textcolor[rgb]{0.00,0.00,0.00}{#1}}
\newcommand{\CharTok}[1]{\textcolor[rgb]{0.31,0.60,0.02}{#1}}
\newcommand{\SpecialCharTok}[1]{\textcolor[rgb]{0.00,0.00,0.00}{#1}}
\newcommand{\StringTok}[1]{\textcolor[rgb]{0.31,0.60,0.02}{#1}}
\newcommand{\VerbatimStringTok}[1]{\textcolor[rgb]{0.31,0.60,0.02}{#1}}
\newcommand{\SpecialStringTok}[1]{\textcolor[rgb]{0.31,0.60,0.02}{#1}}
\newcommand{\ImportTok}[1]{#1}
\newcommand{\CommentTok}[1]{\textcolor[rgb]{0.56,0.35,0.01}{\textit{#1}}}
\newcommand{\DocumentationTok}[1]{\textcolor[rgb]{0.56,0.35,0.01}{\textbf{\textit{#1}}}}
\newcommand{\AnnotationTok}[1]{\textcolor[rgb]{0.56,0.35,0.01}{\textbf{\textit{#1}}}}
\newcommand{\CommentVarTok}[1]{\textcolor[rgb]{0.56,0.35,0.01}{\textbf{\textit{#1}}}}
\newcommand{\OtherTok}[1]{\textcolor[rgb]{0.56,0.35,0.01}{#1}}
\newcommand{\FunctionTok}[1]{\textcolor[rgb]{0.00,0.00,0.00}{#1}}
\newcommand{\VariableTok}[1]{\textcolor[rgb]{0.00,0.00,0.00}{#1}}
\newcommand{\ControlFlowTok}[1]{\textcolor[rgb]{0.13,0.29,0.53}{\textbf{#1}}}
\newcommand{\OperatorTok}[1]{\textcolor[rgb]{0.81,0.36,0.00}{\textbf{#1}}}
\newcommand{\BuiltInTok}[1]{#1}
\newcommand{\ExtensionTok}[1]{#1}
\newcommand{\PreprocessorTok}[1]{\textcolor[rgb]{0.56,0.35,0.01}{\textit{#1}}}
\newcommand{\AttributeTok}[1]{\textcolor[rgb]{0.77,0.63,0.00}{#1}}
\newcommand{\RegionMarkerTok}[1]{#1}
\newcommand{\InformationTok}[1]{\textcolor[rgb]{0.56,0.35,0.01}{\textbf{\textit{#1}}}}
\newcommand{\WarningTok}[1]{\textcolor[rgb]{0.56,0.35,0.01}{\textbf{\textit{#1}}}}
\newcommand{\AlertTok}[1]{\textcolor[rgb]{0.94,0.16,0.16}{#1}}
\newcommand{\ErrorTok}[1]{\textcolor[rgb]{0.64,0.00,0.00}{\textbf{#1}}}
\newcommand{\NormalTok}[1]{#1}
\usepackage{longtable,booktabs}
\usepackage{graphicx,grffile}
\makeatletter
\def\maxwidth{\ifdim\Gin@nat@width>\linewidth\linewidth\else\Gin@nat@width\fi}
\def\maxheight{\ifdim\Gin@nat@height>\textheight\textheight\else\Gin@nat@height\fi}
\makeatother
% Scale images if necessary, so that they will not overflow the page
% margins by default, and it is still possible to overwrite the defaults
% using explicit options in \includegraphics[width, height, ...]{}
\setkeys{Gin}{width=\maxwidth,height=\maxheight,keepaspectratio}
\IfFileExists{parskip.sty}{%
\usepackage{parskip}
}{% else
\setlength{\parindent}{0pt}
\setlength{\parskip}{6pt plus 2pt minus 1pt}
}
\setlength{\emergencystretch}{3em}  % prevent overfull lines
\providecommand{\tightlist}{%
  \setlength{\itemsep}{0pt}\setlength{\parskip}{0pt}}
\setcounter{secnumdepth}{5}
% Redefines (sub)paragraphs to behave more like sections
\ifx\paragraph\undefined\else
\let\oldparagraph\paragraph
\renewcommand{\paragraph}[1]{\oldparagraph{#1}\mbox{}}
\fi
\ifx\subparagraph\undefined\else
\let\oldsubparagraph\subparagraph
\renewcommand{\subparagraph}[1]{\oldsubparagraph{#1}\mbox{}}
\fi

%%% Use protect on footnotes to avoid problems with footnotes in titles
\let\rmarkdownfootnote\footnote%
\def\footnote{\protect\rmarkdownfootnote}

%%% Change title format to be more compact
\usepackage{titling}

% Create subtitle command for use in maketitle
\providecommand{\subtitle}[1]{
  \posttitle{
    \begin{center}\large#1\end{center}
    }
}

\setlength{\droptitle}{-2em}

  \title{Planning Tool Guidance}
    \pretitle{\vspace{\droptitle}\centering\huge}
  \posttitle{\par}
    \author{Project Big Life}
    \preauthor{\centering\large\emph}
  \postauthor{\par}
      \predate{\centering\large\emph}
  \postdate{\par}
    \date{2019-06-10}

\usepackage{booktabs}
\usepackage{amsthm}
\makeatletter
\def\thm@space@setup{%
  \thm@preskip=8pt plus 2pt minus 4pt
  \thm@postskip=\thm@preskip
}
\makeatother

\begin{document}
\maketitle

{
\setcounter{tocdepth}{1}
\tableofcontents
}
\chapter{Welcome to Project Big Life's Planning
Tool}\label{welcome-to-project-big-lifes-planning-tool}

To do: Include motivation: why someone should use the platform (video
and text)

\begin{verbatim}
## PhantomJS not found. You can install it with webshot::install_phantomjs(). If it is installed, please make sure the phantomjs executable can be found via the PATH variable.
\end{verbatim}

\begin{Shaded}
\begin{Highlighting}[]
\CommentTok{#library(rsconnect)}
\CommentTok{#bookdown::publish_book(name = "BATCH_Guidance", account = "rhiannon_roberts22", server = "bookdown.org", }
  \CommentTok{#render = c("server"))}

\CommentTok{#install.packages("testit")}
\CommentTok{#library(testit)}
\CommentTok{#install.packages("contrib.url", repos= "http://cran.us.r-project.org")}
\end{Highlighting}
\end{Shaded}

\chapter{Introduction}\label{introduction}

The Project Big Life Planning Tool was developed in order to support
health practitioners: research, plan, develop, and evaluate
evidence-based health interventions.

For instance Project Big Life Planning Tool helps:

\begin{itemize}
\tightlist
\item
  Public health practitioners: Assess the impact of a preventative
  intervention on a health behaviour.
\item
  Health planners: Assess the need for palliative care.
\end{itemize}

\textbf{What types of questions can it answer?}

\begin{itemize}
\tightlist
\item
  What is the burden of smoking on life expectancy?
\item
  How many deaths would be prevented if everyone met their daily
  excersize requirements?
\end{itemize}

\textbf{How does it work?}

\begin{itemize}
\item
  This tool provides health planners with access to multivariable
  predictive risk algorithms, created and housed by the Project Big Life
  Team.
\item
  The multivariable predictive risk algorithms use distinct
  characteristics and health profiles of groups of people to assess the
  risk of a health outcome (e.g.~Life Expectancy).
\item
  The multivariable predictive risk algorithms are developed and
  validated using routinely collected data by Statistics Canada and
  provincial health agencies, and the algorithms have been published in
  various journals.
\item
  More information about multivariable predictive risk algorithms can be
  found in the key concepts (Chapter \ref{keytopics}).
\end{itemize}

\textbf{Why should I used it?}

\begin{itemize}
\item
  It is \textbf{easy} and \textbf{flexible} to use.

  \begin{itemize}
  \tightlist
  \item
    The user only needs to upload their data and choose which
    calculation to run.
  \item
    It can be used to assess the current or future risk of a health
    outcome.
  \item
    It can be used to assess the effectiveness of different intervention
    scenarios (e.g.~policy) on a health outcome.
  \end{itemize}
\item
  It generates \textbf{accurate} predictions.

  \begin{itemize}
  \tightlist
  \item
    It can be used to accurately assess the risk of a health outcome in
    populations that were not used in its developement, and groups of
    people that account for only a fraction of the population.
  \end{itemize}
\item
  It is \textbf{Private}.

  \begin{itemize}
  \tightlist
  \item
    Uploaded data remains on your computer and is not uploaded or sent
    anywhere.
  \end{itemize}
\end{itemize}

\chapter{Getting Started}\label{getting-started}

To help you get started with Project Big Life's Planning Tool quickly,
we built a Tutorial directly onto the platform.

The tutorial takes you through step-by-step how to use Project Big
Life's Planning Tool. The tutorial will not explain the steps in detail
(Chapter \ref{howto}) nor will it provide reference material (Chapter
\ref{glossary}), but it will give you an understanding of how easy it is
to use the Planning Tool!

\textbf{To access the tutorial}, go onto Project Big Life's Planning
Tool (\url{http://policy.projectbiglife.ca/}) and click on the Tutorial
button \emph{placeholder to insert the picture of the tutorial `button'}
in the top right corner!

\chapter{How To}\label{howto}

These guides will cover similar topics as the tutorials but in greater
detail.

\begin{itemize}
\tightlist
\item
  Import Data
\item
  Filter Data
\item
  Stratify Data
\item
  Calculate Outcomes
\item
  Visualize Data
\item
  Export Data
\item
  Generate Intervention Scenarios
\item
  Resolve Error Messages
\end{itemize}

\section{Import Data}\label{import-data}

\section{Filter Data}\label{filter-data}

\section{Stratify Data}\label{stratify-data}

\begin{itemize}
\tightlist
\item
  Default Stratification
\item
  Custom Stratification Include the coding steps for R/STATA/SAS
\end{itemize}

\section{Calculate Outcomes}\label{calculate-outcomes}

*Note that the larger the file the longer the calculations will take.
Use LE or Risk of death as examples in the how-to

\section{Visualize Data}\label{visualize-data}

\begin{itemize}
\tightlist
\item
  export
\item
  create your own(?)
\end{itemize}

\section{Export Data}\label{export-data}

\section{Generate Intervention
Scenarios}\label{generate-intervention-scenarios}

\begin{itemize}
\tightlist
\item
  Define target population (all or high-risk only), and specify
  intervention effect
\end{itemize}

\section{Resolve Error Messages}\label{resolve-error-messages}

\begin{itemize}
\item
  Out of range
\item
  Sample Size is too small
\end{itemize}

\chapter{Applications}\label{applications}

This chapter provides you with examples of how Project Big Life's
Planning Tool can be used in your day-to-day operations. The examples
will cover: generating a health status report, determining the impact of
a local policy, and determing the impact of a national policy.

\emph{Note to self: All of the data manipulation will occur through R
and then be uploaded to the platform to be run.}

\section{Health Status Report}\label{health-status-report}

In this example we will generate a health status report. This will
include:

\section{Diet}\label{diet}

\section{Transportation}\label{transportation}

\chapter{Key Topics}\label{keytopics}

This section explains some key concepts in Project Big Life's Planning
Tool. This section will explain how it works rather then how to do
things.

\begin{itemize}
\tightlist
\item
  Multivariable Predictive Risk Algorithms
\end{itemize}

\section{Multivariable Predictive Risk
Algorithms}\label{multivariable-predictive-risk-algorithms}

Multivariable predictive risk algorithms predict the future risk of
health outcomes (e.g.~Life Expectancy) for a population using routinely
collected health data.

Multivariable predictive risk algorithms can be used to:

\begin{itemize}
\tightlist
\item
  Project the number of new cases of the health outcome
\item
  Estimate the contribution of specific risk factors of the health
  outcome
\item
  Evaluate effectiveness of health interventions
\item
  Describe the distribution of risk in the population (diffused or
  concentrated)
\end{itemize}

Multivariable predictive risk algorithms are able to assess equity
issues compared to competing population risk methods (e.g.~World Health
Organization Global Burden of Disease).

More information on what multivariable predictive risk algorithms are
and how they can be used can be found the journal article:
\emph{Predictive risk algorithms in a population setting: an overview}
\citep{PoRTover}

\textbf{4.1.1 Development of multivariable predictive risk models}

Data:

\begin{itemize}
\item
  Multivariable predictive risk models are created using routinely
  collected data that includes information about risk factors (exposure)
  and health events (outcomes).
\item
  Data is collected at an individual level through population health
  surveys (e.g.~Canadian Community Health Survey) and administrative
  databases (e.g.~Vital Statistics). Data sources are linked together
  when the individual has given permission too.
\item
  Individuals are followed overtime until the health event (e.g.~death
  or disease) occurs.
\item
  Separate data is collected to create a derivation cohort and
  validation cohort(s).

  \begin{itemize}
  \tightlist
  \item
    Note: The risk factors that are collected are from population health
    surveys and are self-reported; no clinical data (e.g.~blood
    pressure) is collected. Risk factors focus on health behaviours
    (e.g.~smoking) and sociodemographic factors, commonly associated
    with health outcome.
  \end{itemize}
\end{itemize}

\textbf{Model Generation:}

\begin{itemize}
\item
  Multivariable predictive risk models are developed and validated in 4
  stages:

  \begin{itemize}
  \tightlist
  \item
    Model derivation: the predictive risk model is created using data
    from the derivation cohort
  \item
    Model validation: the predictive risk model is applied to the
    validation cohort
  \item
    Final Model Generation: validation and derivation cohorts are
    combined to estimate the final application of the predictive risk
    model
  \item
    Derivation of the application model: creation of a parsimonous
    (fewer predictors) model that maintained discrimination,
    calibration, and overall model performance
  \end{itemize}
\item
  In each stage of the model development and validation, model
  performance is assessed using measures of discrimination and
  calibration.
\end{itemize}

\textbf{4.1.2 Multivariable predictive risk algorithms in Project Big
Life Planning Tool}

\begin{itemize}
\tightlist
\item
  There is currently 1 multivariable predictive risk model is built into
  to Project Big Life planning tool.
\end{itemize}

Title

Outcomes

Information

Mortality Population Risk Tool

5 year risk of death, Life Expectancy, Life years lost

Appendix 1

\section{Cause-Deleted}\label{cause-deleted}

\section{Life Expectancy Calculation}\label{life-expectancy-calculation}

\section{Health Interventions
Scenarios}\label{health-interventions-scenarios}

The Project Big Life Planning Tool is a robust, it allows you to run
many different types of health intervention scenarios.

\chapter{Glossary}\label{glossary}

\textbf{5-year mortality risk}

The probability that an individual will die in the next 5 years.

\textbf{Body Mass Index (BMI)}

A weight-to-height ratio used as an indicator of obesity and
underweight. BMI is calculated by dividing an individual's body weight
in kilograms by the square of height in metres (kg/m2).

\textbf{Burden}

The impact or size of a health problem in an area, measured by cost,
mortality, morbidity or other indicators. The burden of unhealthy
behaviour is calculated by the differences in life expectancy based on
individuals' exposure to four health behavioural risks for poor health
\emph{relative to what for continous outcomes}. - Seven More Years

\textbf{By Row Measures}

When selected, the output `.csv' file will include the result of the
calculation for each row (e.g.~individual) of the dataset.

\textbf{Calibration}

The agreement between predicted risk generated from the model and
observed risk generated from the data.

\textbf{Canadian Community Health Survey}

An annual survey preformed by Statistics Canada that collects
information related to health status, health care utilization and health
determinants for the Canadian population. Details about the survey can
be found on Statistic Canada website
(\url{http://www23.statcan.gc.ca/imdb/p2SV.pl?Function=getSurvey\&Id=795204}).

\textbf{Discrimination}

The ability of the model to differentiate between high risk individuals
and low risk individuals.

\textbf{Error Message}

Error messages will occur when variables that are \textbf{``Required for
Calculation''} are missing in the data. If the entire column for the
variable is missing then the calculation cannot be performed on the
data. If there are missing row entries for the variable then the entire
row will not be used in the calculation.

\textbf{Filter}

Chooses part of your dataset for analysis. If you filter on
`Sex' and then `Male', calculations will only be performed on
individuals that are `Male' and `Females' will be excluded. For example,
when calculating Life Expectancy on the filter variable `Sex' then
`Male' there will be a Life Expectancy estimate for `Males' and
\emph{no} Life Expectancy estimate for `Females'.

\textbf{Health Behaviour}

Actions people do that may affect their health, positively or
negatively. Health behaviours are among the determinants of health and
are influenced by the social, cultural and physical environments in
which people live and work.\citep{StatsCan2010} They are also shaped by
individual choices and external constraints.\citep{StatsCan2010} The
four health behaviours of \textbf{smoking, alcohol consumption, diet,}
and \textbf{physical activity} are specified in Project Big Life's
planning tool.

\textbf{Ignored Variables}

Are not included in the calculation. It does not matter if your dataset
includes these variables or not. Ignored variables can used for filter
and stratification.

\textbf{Life Expectancy (LE)}

Life expectancy is a calculation of how long a person or
population would be expected to live, on average, given unchanging risk
of death from a specific point in time.

\textbf{Metabolic Equivalent of Task (MET)}

The metabolic equivalent of task (MET) is a measure of the rate of
energy expenditure from an activity; a measure of calories burned by
type, duration and frequency of physical activity. The reference value
of 1 MET is defined as the energy expediture rate at rest which is equal
to 1kcal/kg/day.

\textbf{Recommend for calculation}

Variables that are included in the calculation but not necessary for the
calculation to run. Rather these variables increase the accuracy of the
results.

\textbf{Required for calculation}

Variables that are included in the calculations and are necessary for
the calculation to run. If a dataset does not have these variables then
the calculation will not run.

\textbf{Socioeconomic Position}

People in poorer socioeconomic circumstances generally have poorer
health. Deprivation measures identify those who experience material or
social disadvantage compared to others in their community.\emph{3 - Link
no longer exists} The Deprivation Index for Health in Canada developed
by the Institut national desanté publique du Québec
(INSPQ)\citep{INSPQ2000} is used in this plannning tool. The index
includes education, employment and income as measures of material
deprivation; and single-parent families, living alone, or being
divorced, widowed or separated as measures of social deprivation. The
deprivation index was used to assign geographical areas into
socioeconomic position groups (low, middle and high) based on material
and social quintiles. High-deprivation neighbourhoods were those in the
top two quintiles for both social and material deprivation.
Low-deprivation neighbourhoods were those in the bottom two quintiles.

\textbf{Stratification}

The seperation of data into smaller strata (levels or
classes which individuals are assigned too). If the variable `Sex' is
stratified it creates two strata: `Male' and `Female'. Calculations are
performed on each strata (level or class) and the outcome will be
specific to that strata. For example, when calculating Life Expectancy
on the stratified variable `Sex' there will be a Life Expectancy
estimate for `Males' and a different Life Expectancy estimate for
`Females'.

\textbf{Summary Measures}

When selected, the output .csv file will include the calculation result
for the entire population of the dataset.

\textbf{Warning Message}

Warning messages will occur when variables that are
\textbf{``Recommended for Calculation''} are missing in the data. If the
entire column for the variable is missing the calculation will still be
performed on the data. If there are missing row entries for the variable
the row will still be used in the calculation.





















\appendix


\chapter{Mortality Population Risk Tool
(MPoRT)}\label{mortality-population-risk-tool-mport}

\textbf{Outcomes: 5-yr risk of death, Life Expectancy, Cause-deleted
Life Expectancy}

\textbf{Calculations}

Using MPoRT you are able to calculate:

\begin{itemize}
\tightlist
\item
  5 year mortality risk
\item
  Number of deaths
\item
  Life Expectancy
\item
  Cause-deleted deaths and life expectancy
\item
  Burden of health behaviour in deaths and on life expectancy
\end{itemize}

\textbf{Types of Questions}

\begin{itemize}
\tightlist
\item
  What is the burden of smoking on life expectancy?
\item
  How many deaths would be prevented if everyone met their daily
  excersize requirements?
\end{itemize}

\textbf{Description}: A multivariable predictive risk model that
estimates the future risk of all-cause death in Canada. It adjusts for
health behaviours: smoking, unhealthy alcohol consumption, poor diet,
and physical inactivity, and a wide range of other risk factors.

Versions of MPoRT have been developed since 2012 and used in various
studies. Each version of MPoRT (v1.0, v1.2, v2.0) used the Ontario
subset of the Canadian Community Health Survey (CCHS) for development
and the survey respondents were linked to personal death records. In
later versions of MPoRT (v1.2, v2.0) the following changes were made:,
(a) algorithm variables were adjusted to improve predictions, and (b)
the algorithms were validated using: the Ontario subset of CCHS of the
years that were not used in development and the National CCHS dataset
(excluding Ontario).

\textbf{MPoRTv1.0} Was used in the ``Seven More Years'' report, a joint
report with Public Health Ontario and IC/ES
(\url{https://www.ices.on.ca/Publications/Atlases-and-Reports/2012/Seven-More-Years}).
In summary, the algorithm estimated the risk of death associated with
health behaviours: smoking, unhealthy alcohol consumption, poor diet,
physical inactivity and stress. There were approximately 550,000
person-years of follow up and over 6000 deaths in the development
dataset. The algorthim used categorical predictor variables for health
behaviours and sociodemographic factors.

\textbf{MPoRTv1.2} Was published in PLoS
(\url{https://journals.plos.org/plosmedicine/article?id=10.1371/journal.pmed.1002082}).
In summary, the algorithm estimated the risk of death associated with
health behaviours: smoking, unhealthy alcohol consumption, poor diet,
and physical inactivity (stress was removed due to its low prediction
ability). There were approximately 1 million person-years of follow up
and over 9000 deaths in the development and validation datasets. The
algorithm used multiple continous predictor variables, and added chronic
disease predictor variables and interaction terms.

\textbf{MPoRTv2.0 - The version used in Project Big Life's Planning
Tool} This version of MPoRT has not yet been published.

\emph{Development}: This predictive risk model was developed using
Ontario subsets of the 2001 to 2008 CCHS and participants were linked to
personal health records. There were approximately 1.3 million
person-years of follow-up and over 15,000 deaths in the developmental
dataset.

\emph{Validation}: This predictive risk model was validated using three
different datasets: Ontario subset of the 2009 to 2012 CCHS, National
dataset (except Ontario) of the 2003 to 2008 CCHS, and the National
dataset of the 2000 and 2005 National Health Interview Survey in the
United States of America. In all validation datasets individuals were
linked to personal health records.

\emph{Parameters}: The parameters used in this predictive risk model
are:

Category

Variable

Scale

Description

Demographic

Age*

Continous

5 knot spline. Valid range 20 to 102

Sex

Dichotomous

Stratified Female/Male

Health Behaviour

Pack years of smoking

Continous

3 knot spline. Valid range: 0 to 78 (Female), 0 to 112.5 (Male)

Smoking Status

Categorical

Non-smoker

Current Smoker

Former Smoker \textless{}= 5 years

Former \textgreater{} 5 years

Alcohol (number of drinks per week)

Continous

4 knot spline (Females) and 3 knot spline (Males). Valid range: 0 to 25
(Female), 0 to 50 (Male)

Former/non-drinker

Dichotomous

Yes/No

Simplified diet score

Continous

3 knot spline. Valid range: -18.9 to 20.7 (Female), -16.8 to 18.4 (Male)

Leisure physical activity (MET)

Continous

3 knot spline. Valid range: 0 to 12.4 (Female), 0 to 16 (Male)

Socio-demographic

Ethnicity

Categorical

White

Black

Chinese

Arab; South Asian; West Asian

Filipino; Japanese; Korean; Southeach Asian

Other; Indigenous; Latin American; Multiple origin; unknown

Immigrant

Dichotomous

Yes/No

Fraction of lifetime in Canada

Continous

3 knot spline\textsuperscript{†}. Valid range: 0 to 1

Education

Categorical

Less than secondary

Secondary School Graduation

Some Post-Secondary

Post-Secondary Graduation

Neighbourhood social and material deprivation

Ordinal

Low (1st or 2nd quantile

High (4th or 5th quantile)

Moderate (all others)

Chronic Conditions

Diabetes

Dichotomous

Yes/No

High Blood Pressure

Dichotomous

Yes/No

Chronic Respiratory Disease

Dichotomous

Yes/No

Mood Disorder

Dichotomous

Yes/No

Cancer

Dichotomous

Yes/No

Dementia

Dichotomous

Yes/No

Heart Disease

Dichotomous

Yes/No

Stroke

Dichotomous

Yes/No

Epilepsy

Dichotomous

Yes/No\textsuperscript{‡}

BMI

Continous

3 knot spline. Valid range: 8.9 to 47.2 (Female), 8.6 to 43.7 (Male)

* Age interaction included for all variables exept immigrant, fraction
of time in Canada, and ethnicity † Excluded in the male model, remains
in the female model ‡ Excluded in the female model, remains in the male
model

\bibliography{book.bib,packages.bib}


\end{document}
